\chapter{Conclusions and Future Work}\label{chp:conclusions-and-future-work}
\section{Conclusions}\label{sec:conlclusions}
The results of the analysis provided in the chapter \ref{chp:results-and-observations}, provide a conclusion that it is indeed possible to achieve a high level of accuracy when attempting to predict whether a given source code file is likely to contain a bug from the static code metrics. 

Previous research in the area of automatic prediction of bugs \cite{autoDetectionOfPerfBugs} achieved accuracy of 85\%, however, it focused solely on a specific type of bugs related solely to the performance of a given application.
The underlying case study, not only broadened the approach by extending the experiment of all types of bugs, but it also raised the accuracy of prediction to, at its most optimized parameters, approximately 99\%.

\section{Future Work}\label{sec:future-work}
The first potential future research could focus on confirming the underlying findings in monolithic applications as opposed to those following microservice architecture.

Following on from that it is envisioned that another branch of research could focus on the impact, or lack thereof, the application architecture can have on the presence of bugs, and the ratio of bug vs. non-bug development, between microservice and monolith architectures.
