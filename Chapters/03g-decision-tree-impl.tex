\subsubsection{Implementing Decision Tree classification}
The decision tree model is first executed on the unscaled downsampled and upsampled datasets in order to establish baseline for both datasets when examining the results of the scaled models.

The execution was conducted in the following steps:\
\begin{itemize}
    \item the dataset under analysis is split into the features used for predictions and the target feature to be predicted, $X$ and $y$ respectively.
    \item the $X$ and $y$ are split into the \texttt{train} and \texttt{test} sets. The \texttt{train} subset is used to train the model and the \texttt{test} subset is used to then verify the predictions made using data points previously unseen by the model in question. The split is $75\%$ data points dedicated to the \texttt{train} subset and $25\%$ to the \texttt{test} subset.
    \item the decision tree model is instantiated
    \item the model is trained by fitting the training set 
    \item new unseen data points are simulated using the test subset
    \item Sci-kit Learn's \texttt{accuracy\_score} function is utilized to determine the accuracy of the prediction made using the simulated new data points and the features of the test subset. The result is presented as a floating point number between 0 and 1, with values closer to 1 representing higher accuracy of the model
    \item the result is multiplied by 100 in order to convert it to a $\%$ value.
\end{itemize}

The same steps are conducted for all of the scaled datasets and the final accuracy score, as per method described is presented.
