\chapter{Background Research}
Microservices architecture took the world by the storm from as early as 2011 \cite{fowlerOnMicroservices}, with major companies, such as Facebook, Amazon or Netflix, utilizing this architectural style with great success. 
Building upon the definition by Dragoni et al \cite{dragoniOnMs} and Adrian Cockcroft \cite{understandingOfMicroservices} microservices can be described as pieces of software that are highly cohesive, decoupled from one another and which can be developed, deployed and executed independently. \cite{dragoniOnMs} and \cite{understandingOfMicroservices} also agree that all microservices should have just one objective that is performed well, however, a single microservice isn't very useful on its own. Only when a system is formed from individual microservices does the usefulness of said architecture materializes.

In any software product, regardless of its architecture or deployment principles, code defects also referred to as bugs, will inevitably occur. Currently, bug detection is still primarily a human-driven process carried out by a mixture of manual and automated tests in the hope of identifying bugs as much as in verifying that the tested parts of the larger system are behaving correctly. In the context of distributed systems, such as applications written using microservice architecture paradigm, it is important to realize that testing should cover both the individual microservices as well as the entirety of the system.

The entirety of the system, rather than the sum of its parts, should be tested as a bug in a single microservice can potentially affect a larger ecosystem. To give an example, imagine a microservice responsible for authentication "login" and another for authorization, "auth", to a shopping site generating revenue directly from customer purchases, "Foo". It should be noted that Foo constitutes of various microservices apart from the ones already mentioned. Now imagine that the customers are seeing a blank page when attempting to log into their accounts - making all purchasing ability unavailable. Foo is a distributed system and a few microservices or pieces of the infrastructure could be responsible for the bug occurring, making the bug identification mechanisms for the distributed system , or development of an automated test suite for the entirety of the system a priority, as even in our simple scenario we have 2 likely candidates responsible for the failure. 

However, especially in the context of CD environments, where speed of delivery is crucial, it is inconceivable to have automated tests extending to the entirety of a system as the delivery time would increase in proportion to the number of test scenarios possible within the system\cite{softwareTestingChallenges}, while at the same time, it remains crucial to preferably prevent, or at least quickly identify bugs. 

The above situation is precisely the reason why additional research into the automatic prediction of bugs is being proposed as the preferable solution to bug identification - it would prevent the Foo customers from experiencing the problem, and prevent damage to the reputation of the business behind it.

It has already been stipulated in studies\cite{autoDetectionOfPerfBugs}, as well as circulating in the software development industry media\cite{costOfBugInRelationToDevelopmentPhase}, that the impact of a bug rises proportionally to the stage of development it has been identified in.
Given the above scenario as well as previous research outlined\cite{bugDetectionInParticleSwarm}, it is important to devise a strategy to prevent bugs in components of the system and it is equally important to prevent defects when individual system components are integrated. Additionally the larger the system the more difficult it becomes to identify the area in which defects originated in\cite{Zhou_2012_whereShouldBugsBeFixed}, regardless of the software architecture implemented, as with the microservices ecosystem the total number of lines of code (LoC) would be the total of the individual components, while for the monolithic applications it would simply be the total number of LoC in the single module. 

Furthermore, taking into the account that it is estimated that software maintenance takes up as much as two-thirds of a software's life cycle\cite{duplicateBugDetection} with some studies raising that number to approximately 90\%\cite{reopenedBugsAndMaintenence}, bug detection becomes imperative and it should be any organization's priority to attempt to detect bugs as early as possible, the earliest possible being, during a given feature's development before it's been released. However, the rise of Continuous Deployment practices, stating that each piece of software integrated with main product code base has to be automatically released into production environment complicates the landscape as the code base of any component and conversely, the bug count has potential to change rapidly even during a single day.

Previous research suggests that it is possible to predict an occurrence of bugs using currently known Machine Learning algorithms\cite{autoDetectionOfPerfBugs}, however, the parameters used for building prediction model in the cited research is different than ones propose below. Furthermore, the effect, if any, of the software architecture, or the deployment process implemented in the system has not been represented in any previous research encountered. It is also relevant to point out that Tsakiltsidis et all\cite{autoDetectionOfPerfBugs} narrowed down the focus of their research solely to detection of performance bugs while no such restrictions are enforced in this proposal.

In this research, the focus will be placed on the broader scope as well as a different set of parameters to build a prediction model: 
\begin{enumerate}
\item amount of existing tech debt
\item amount or lack of code coverage
\item level or code coverage present (unit, integration within a given application module, integration between modules)
\item level of code duplication
\item number of code deploys per a given period (hour, day, week, etc)
\item application architecture
\item deployment strategy, continuous vs manual
\end{enumerate}
The aim is for the prediction metrics to be obtained from static code analysis tools making it suitable for the larger code bases and systems by mitigating the necessity to build large test environments to perform code execution to obtain results. As previously established, the bigger the system, distributed or not, the longer it would take to test its integration, however, static code analysis does not suffer from the combinatorial compounding of the test scenarios available as it doesn't execute any code as part of its operation.

Above is not a definite list and it is probable that as research progresses it may grow to include additional factors should they enhance the accuracy of the prediction model. It remains to be established if any of the below factors have any effect on the bug prediction accuracy:
\begin{enumerate}
\item application language
\item presence of peer reviews of the code to be integrated
\item development style utilized, e.g. single programmer vs. pair-programming
\item channel of communication between system components
\item \label{fileAge} time lapsed since last file modification
\item \label{developerMastery}level of developer's experience with a given application
\end{enumerate}