\chapter{Introduction}\label{chp:intro}
\chaptermark{Introduction}

\section{Research Paper Purpose}
This dissertation is intended to answer the following research questions:
\begin{itemize}
\item Explore the correlation between static code metrics and the probability of a code defect occurring
\item Research the possibility of establishing a prediction model based on static code metrics to the probability of a code defect occurring 
% \item Exploring the effects of using Microservices architecture and Continuous Deployment strategy on the established prediction model
\end{itemize}
\section{Research Paper Structure}
It is divided into 5 chapters, including the current one. Chapter \ref{chp:background-research} will discuss the background for the given research as well as any previous research conducted in the same space and context. 

Chapter \ref{chp:design-and-impl} focuses on the solution's architecture, discusses any predictive machine learning models used as well as delves into the solution implementation in detail. All observations made during the development and research will be the focus of chapter \ref{chp:results-and-observations}, detailing all results obtained.

Finally, chapter \ref{chp:conclusions-and-future-work} provides a conclusion to the paper as well as making suggestions as to what possible future research may be conducted in the same research space.

\section{Glossary of terms}
\begin{itemize}
    \item data frame - also spelt as dataframe, is a two-dimensional data construct where the data is aligned in a tabular fashion in rows and columns
    \item column, feature, attribute - used interchangeably to describe a given series of data in the data frame
    \item heatmap - a type of graph used to visualize the correlation between attributes in a given data frame
    \item bug - referred to in general as a fault in the software
    \item CD - refers to Continuous Deployment, a principle of software development projects where all code changes are periodically made available, or deployed, to the target platform for any users to make use of. 
    \item CI - refers to Continuous Integration, a software development principle where all code is continuously integrated. Note difference from CD - in CI the code is not continuously deployed to the target platform
    \item API - Application programming interface, a set of functions and procedures allowing the communication with a given application, or set of applications, in order to instruct the application to perform an or any number of actions.
    \item REST - stands for Representational State Transfer. It is a method of providing a standard communication method between computer systems on the web.
    \item REST endpoint - a reference to a URL that accepts REST requests over the web
\end{itemize}